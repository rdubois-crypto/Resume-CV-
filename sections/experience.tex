
%====================
% EXPERIENCE A
%====================
\subsection{{Senior Cryptographer / Ledger Innovation Team \hfill 04/2022 --- Aujourd'hui}}
\subtext{Ledger \hfill Paris, France}
\begin{zitemize} 
\item Veille technologique: ZKP, Decentralized Identity, Attribute-Based crypto
\item Développement de PoC au sein de l'équipe innovation
\item Soutien à l'intégration du framework Starknet
\end{zitemize}

\vskip+0.3cm
%====================
% EXPERIENCE B
%====================
%====================
% EXPERIENCE C
%====================
\subsection{{Expert Cryptologue / Laboratoire Chiffre \hfill 2019 --- }}
\subtext{Thales \hfill Gennevilliers, France}

\vskip+0.5cm
Le périmètre de mon poste au sein du Laboratoire chiffre de THALES s'est enrichi graduellement avec deux promotions au sein de la filière expertise. La séparation des tâches est plus intriquée que comme restitutée par commodité ici. 

\begin{zitemize}
\item {\bf Référent crypto sur plusieurs projets Défense.}
Ce terme propre  à THALES de "référent crypto" consiste en un mélange d'expertise technique (correspondant identifié vis-à-vis du client sur les aspects expertise crypto) et managementde projet transverse. Pour les projets type "réalisation de composants", cela signifie s'assurer de la conformité de la réalisation de services cryptographiques rendus par un composant conformément aux spécifications de besoins, en compatibilité avec des règles et normes (règles FIPS pour des composants exports, règles RGS issues de l'ANSSI pour les composants souverains). Management transverse de cryptologues juniors (respect des coûts délais et coaching). Pour les projets type "intégration de service crypto dans des systèmes satellitaires ou radio", assurer
la coordination entre l'ingénierie système, les équipes de développement et les experts clients DGA/ESA
\item {\bf Rôle officieux de Responsable innovation du Laboratoire.}

\end{zitemize}

%====================
% EXPERIENCE D
%====================
\subsection{{Spécialiste Cryptologue / Laboratoire Chiffre \hfill 01/2011 --- 01/2019}}
\subtext{Thales Communications \hfill Gennevilliers, France}
\begin{zitemize}
\item {\bf Responsable chargé d'affaire et coordinateur technique du projet de recherche BEST (Broad-cast Encryption for Secure Telecommunications).} Encadrement industriel d'une thèse et de stagiaires. Le but de ce projet de 4 ans était de concevoir un protocole de broadcast encryption adapté à une faible bande passante pour beaucoup d'utilisateurs. Le projet réunissait un consortium de 2 industriels (THALES, NAGRA) une PME (CE) et des académiques (ENS Ulm, Paris VIII). Mon rôle en externe THALES était d'assurer la coordination technique du consortium sur tout le projet, transmettre les jalons en respectant les délais à l'ANR. En interne THALES j'assurais le suivi des coûts, encadrais en moyenne deux personnes sur le sujet et contribuais techniquement aux activités de R\&D. J'ai contribué à des optimisations algorithmiques (Publication 4.) et la rédaction d'un document de référence (cf Publication 5.) dans une approche similaire aux documents normatifs type ANSIX9.62 (document de description des algorithmes qu'appellent les PKCS).
 
\item {\bf Tuteur, partenaire académique.} Encadrement industriel d'une thèse en MPC. Tuteur d'une douzaine de stages de fin de master. Je place la recherche académique en très haute estime et considère qu'une entreprise doit s'appuyer sur ses publications scientifiques plutôt que sa marque pour revendiquer une légitimité.  
\end{zitemize}

%====================
% EXPERIENCE E
%====================

%====================
% EXPERIENCE D
%====================
\subsection{{Ingénieur Cryptologue / Laboratoire Chiffre \hfill 03/2005  --- 01/2011}}
\subtext{Thales Communications \hfill Colombes, France}
\begin{zitemize}
\item {\bf Ingénierie Système.} Spécification de composants et systèmes développés dans le domaine civil, militaire et export. THALES réalise des systèmes radios et satellitaires. J'ai contribué à spécifier dans des trames (systèmes CONTACT, GALILEO) la réalisation de service cryptographiques (chiffrement, intégrité, authentification) tout en respectant des contraintes (bande passante, latence, menaces identifiées).
\item {\bf Soutien à l'intégration d'algorithmes et de mécanismes cryptographiques.} Développement en C et assembleur d'algorithmes cryptographiques (chiffrerement, signature, authentification). J'ai eu l'opportunité de développer les algorithmes suivants pour le civil : RC4, AES,ECDSA, ECKCDSA, SHA, RSA, MIKKEY-SAKKE (protocole à base de cou-
plage sur courbe elliptique), modes opératoires (ECB, OFB, CBC, GCM, CTR). Tous
ces algorithmes trouvent leur équivalent (et plus) dans le monde Défense, j'en ai
implémenté une vingtaine pour des modèles de référence comme pour du code opérationnel.
Les implémentations doivent répondre à des contraintes d'espace et de débit.
\item {\bf Recherche.} Publications internationales et dépôts de brevets
\item {Enseignement.} Enseignement d'un module de cryptographie à l'université Paris XIII en 2008 et 2009.
\end{zitemize}
%\subsection{{ROLE / PROJECT E \hfill MMM YYYY --- MMM YYYY}}
%\subtext{company E \hfill somewhere, state}
%\begin{zitemize}
%\item In lobortis libero consectetur eros vehicula, vel pellentesque quam fringilla.
%\item Ut malesuada purus at mi placerat dapibus.
%\item Suspendisse finibus massa eu nisi dictum, a imperdiet tellus convallis.
%\item Nam feugiat erat vestibulum lacus feugiat, efficitur gravida nunc imperdiet.
%\item Morbi porta lacus vitae augue luctus, a rhoncus est sagittis.
%\end{zitemize}
\subsection{{Stagiaire Cryptologue / Laboratoire Chiffre \hfill 04/2004  --- 10/2004}}
\subtext{Thales Communications \hfill Colombes, France}
\begin{zitemize}
\item \'Etude des protocoles de broadcast encryption
\item Analyse de la faisabilité : simulations des meilleurs protocoles en C++
\item Dépôt d'un brevet à l'issue du stage
\end{zitemize}


