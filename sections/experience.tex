
%====================
% EXPERIENCE A
%====================
\subsection{{Senior Cryptographer / Ledger Innovation Team \hfill 04/2022 --- Aujourd'hui}}
\subtext{Ledger \hfill Paris, France}
\begin{zitemize} 
\item Technology monitoring: ZKP, Decentralized Identity, Attribute-Based crypto
\item PoC development
\item Support to starknet integration
\end{zitemize}

\vskip+0.3cm
%====================
% EXPERIENCE B
%====================
%====================
% EXPERIENCE C
%====================
\subsection{{Expert Cryptologist / Cryptography Lab \hfill 2019 - Present }}
\subtext{Thales \hfill Gennevilliers, France}

\vskip+0.5cm
The scope of my role within Thales's Cryptography Lab gradually expanded through two promotions within the expertise track. The division of tasks is more intricate than conveniently summarized here.

\begin{zitemize}
\item {\bf Crypto Referent for Multiple Defense Projects: }
This THALES-specific term "crypto referent" entails a blend of technical expertise (being the identified contact for clients regarding cryptographic expertise) and cross-project management. For projects involving "component realization," it involves ensuring the compliance of cryptographic services provided by a component with specified requirements, while adhering to rules and standards (FIPS rules for export components, RGS rules from ANSSI for sovereign components). Cross-management of junior cryptologists (ensuring cost and deadline adherence, and providing coaching). For projects involving the "integration of cryptographic services into satellite or radio systems," it entails coordinating between system engineering, development teams, and client experts from DGA/ESA.
\item {\bf Unofficial Role as Lab Innovation Lead.}

\end{zitemize}

%====================
% EXPERIENCE D
%====================
\subsection{{Cryptologist Specialist / Cryptography Lab\hfill 01/2011 --- 01/2019}}
\subtext{Thales Communications \hfill Gennevilliers, France}
\begin{zitemize}
\item {\bf Business Manager and Technical Coordinator for the BEST Research Project (Broadcast Encryption for Secure Telecommunications).}  Industrial supervision of a thesis and interns. The goal of this 4-year project was to design a broadcast encryption protocol suitable for low bandwidth and a large number of users. The project brought together a consortium of 2 industrial partners (THALES, NAGRA), an SME (CE), and academics (ENS Ulm, Paris VIII). My external role at THALES involved overseeing the consortium's technical coordination throughout the project and delivering milestones to ANR on time. Internally at THALES, I managed costs, supervised an average of two people on the topic, and contributed technically to R\&D activities. I contributed to algorithmic optimizations (Publication 4) and the creation of a reference document (cf Publication 5) in an approach similar to normative documents like ANSIX9.62 (algorithm description document used by PKCS).
 
\item {\bf Academic Partner and Mentor.} Industrial supervision of an MPC thesis. Mentor for a dozen end-of-master internships. I hold academic research in high regard and believe that a company should rely on its scientific publications rather than just its brand to establish legitimacy.
\end{zitemize}

%====================
% EXPERIENCE E
%====================

%====================
% EXPERIENCE D
%====================
\subsection{{Ingénieur Cryptologue / Laboratoire Chiffre \hfill 03/2005  --- 01/2011}}
\subtext{Thales Communications \hfill Colombes, France}
\begin{zitemize}
\item {\bf Systems Engineering.} Specification of components and systems developed for civilian, military, and export domains. THALES creates radio and satellite systems. I contributed to specifying the implementation of cryptographic services (encryption, integrity, authentication) in protocols (CONTACT, GALILEO systems) while meeting constraints (bandwidth, latency, identified threats).

\item {\bf Support for Integration of Cryptographic Algorithms and Mechanisms.} Development in C and assembly of cryptographic algorithms (encryption, signature, authentication). I had the opportunity to develop algorithms including RC4, AES, ECDSA, ECKCDSA, SHA, RSA, MIKKEY-SAKKE (protocol based on elliptic curve pairing), modes of operation (ECB, OFB, CBC, GCM, CTR). All these algorithms have counterparts (and more) in the Defense realm; I implemented around twenty for reference models as well as operational code. Implementations must meet memory and throughput constraints.

\item {\bf Research.} International publications and patent applications.
\item {Teaching.} Taught a cryptography module at the University of Paris XIII in 2008 and 2009.
\end{zitemize}
%\subsection{{ROLE / PROJECT E \hfill MMM YYYY --- MMM YYYY}}
%\subtext{company E \hfill somewhere, state}
%\begin{zitemize}
%\item In lobortis libero consectetur eros vehicula, vel pellentesque quam fringilla.
%\item Ut malesuada purus at mi placerat dapibus.
%\item Suspendisse finibus massa eu nisi dictum, a imperdiet tellus convallis.
%\item Nam feugiat erat vestibulum lacus feugiat, efficitur gravida nunc imperdiet.
%\item Morbi porta lacus vitae augue luctus, a rhoncus est sagittis.
%\end{zitemize}
\subsection{{Cryptology Intern / Cryptography Lab  \hfill 04/2004  --- 10/2004}}
\subtext{Thales Communications \hfill Colombes, France}
\begin{zitemize}
\item Study of broadcast encryption protocols.
\item Feasibility analysis: simulations of the best protocols in C++.
\item Submission of a patent at the end of the internship.
\end{zitemize}


